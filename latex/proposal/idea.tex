\documentclass[10pt]{article}
\usepackage[final]{graphicx}
\usepackage{color}

%\documentclass[11pt]{article}

\usepackage{fullpage,
amsfonts,amssymb,
amsmath,
url,graphics,subfig,
cite,
calc,
psfrag, bm,
amsthm, 
paralist}

%\renewcommand{\textwidth}{5.5in}

%---- Some math. macro
\newcommand\infsum[1][n]{\ensuremath{\sum_{#1=-\infty}^\infty}}

% Here's the definition of Sb, stolen from amstex
    \makeatletter
    \def\multilimits@{\bgroup
  \Let@
  \restore@math@cr
  \default@tag
 \baselineskip\fontdimen10 \scriptfont\tw@
 \advance\baselineskip\fontdimen12 \scriptfont\tw@
 \lineskip\thr@@\fontdimen8 \scriptfont\thr@@
 \lineskiplimit\lineskip
 \vbox\bgroup\ialign\bgroup\hfil$\m@th\scriptstyle{##}$\hfil\crcr}
    \def\Sb{_\multilimits@}
    \def\endSb{\crcr\egroup\egroup\egroup}
\makeatother

\newtheoremstyle{t}         %name
    {\baselineskip}{2\topsep}      %space above and below
    {\rm}                   %Body font
    {0pt}{\bfseries}  %Heading indent and font
    {}                      %after heading
    { }                      %head after space
    {\thmname{#1}\thmnumber{#2}.}

\theoremstyle{t}
\newtheorem{q}{Q}
\parindent=0pt 

\begin{document}

\setcounter{page}{1}
\linespread{1.1}
\normalsize

\begin{center} \Large {\bf
Opinion dynamics in the presence of reluctant agents: \\ and robustifying BitCoin? \\ --- a project for ECS289F ---} \vspace{.5cm}

{\large Hoi-To Wai, Christopher Patton \\

\today}

\end{center}
\vspace{0.3cm}


 \begin{abstract}
This note is a summary of the rudimentary ideas I have for ECS289F's project. Specifically, I will propose a model for the opinion dynamics with reluctant agents, i.e., agents who are reluctant to blend their idea with his/her neighbors. Explorations into the Bitcoin attack model will also be investigated. (For the references cited here, I believe that better references must exists, but I will need to spend more time on mining them.)
\end{abstract}


%-----------------------------------------------------------------------------
%\vspace{0.5cm}

\section{Models for opinon dynamics}
Let $G = (V,E)$ denotes an undirected graph. \vspace{.2cm}


%We shall first discuss the continuous opinion dynamics model. 
We first discuss the continuous opinion dynamics model proposed by DeGroot in 1970s \cite{Degroot_74}. Here, let ${\bm w}_i^{(0)} \in \mathbb{R}^L$ denotes the initial opinion held by the $i$th agent. As time goes by, the opinion of the $i$th agent evolves as:
\begin{equation}\label{eq:op}
{\bm w}_i^{(k)} = \sum_{j \in \mathcal{N}_i} P_{ij} {\bm w}_j^{(k-1)}. 
\end{equation}
Here, $P_{ij}$ refers to the $(i,j)$th entry of a mixing matrix ${\bm P}$, which is right-stochastic such that ${\bm P} {\bf 1} = {\bf 1}$. \vspace{.2cm}

We are interested in the case when ${\bm P}$ is also left stochastic, i.e., we have ${\bf 1}^T {\bm P} = {\bf 1}^T$. Under the mild assumption that the network is connected, i.e., there exists a path connecting any two nodes $i,j$ in the graph. A direct application of the Perron-Frobenius theorem shows that:
\[
\lim_{k \rightarrow \infty} {\bm w}_i^{(k)} = \frac{1}{|V|} \sum_{j \in V} {\bm w}_j^{(0)},~\forall~i \in V.
\]
Furthermore, the convergence speed is shown to be geometric in terms of the 2nd largest eigenvalue of ${\bm P}$.  From this observation, we predict that the social opinion can be able to reach a consensus. As a generalization, the same conclusion holds if ${\bm P}$ depends on the time index $k$, i.e., the mixing matrix is time variant. \vspace{.2cm}

To provide some example regarding this generalization, we can consider a gossip exchange model where the update is asynchronous. At each time step, an agent $i$ wakes up randomly and selects one of its neighbor $j$, also randomly. The two agents then exchange their opinion through averaging:
\[
{\bm w}_i^{(k)} = {\bm w}_j^{(k)} = \frac{{\bm w}_i^{(k-1)} + {\bm w}_j^{(k-1)} }{2},~{\bm w}_p^{(k)} = {\bm w}_p^{(k-1)},~{\rm if}~k \neq i, k \neq j.
\] 
One can verify that the above opinion dynamics is a special case of \eqref{eq:op} with ${\bm P}^{(k)} = {\bf I} - 0.5({\bf e}_i + {\bf e}_j)({\bf e}_i + {\bf e}_j)^T$. Thus, using the previous observation, the opinions of agents also converge to the average. Notice that this is a realistic model at which conversation between agents occurs randomly. \vspace{.2cm}

We now propose a novel model with reluctant agents\footnote{Be reminded that the model proposed below is subjected to change, whenever a better model is derived.}. In this model, we assume that there are three types of agents in the network, namely, normal, reluctant and stubborn, denoted by $V_n, V_r, V_s$, respectively. Each type of the three agents follow a different rule to update his/her opinion. The \emph{normal} agents update their opinion according to \eqref{eq:op}; the \emph{stubborn} agents do not update their opinion at all, i.e.,
\[
{\bm w}_i^{(k)} = {\bm w}_i^{(k-1)},~\forall~k,~i \in V_s.
\]
For the \emph{reluctant} agents, they update their opinion at a slower rate than the others; if a new opinion is presented to them during the adaptation, they will start to adapt the new opinion and stop adapting to the previous one. To describe this dynamic, let  $\tau_i $ be the time required for agent $i$ to adapt to the new opinion and we define:
\[
\hat{\bm w}_i^{(k)} = \sum_{j \in \mathcal{N}_i} P_{ij}^{(k)} {\bm w}_j^{(k-1)},
\]
\[
c_i^{(k)} = \begin{cases}
1 &,~{\rm if}~P_{ij}^{(k)} \neq 0,~\text{for some}~j \in V. \\
\min\{ c_i^{(k-1)} + 1, \tau_i \} &,~{\rm otherwise}.
\end{cases},~\bar{\bm w}_i^{(k)} = 
\begin{cases}
{\bm w}_i^{(k-1)} &,~{\rm if}~P_{ij}^{(k)} \neq 0,~\text{for some}~j \in V. \\
\bar{\bm w}_i^{(k-1)} &,~{\rm otherwise}.
\end{cases}
\]
We can interpret $c_i^{(k)}$ as a counter that monitors if agent $i$ is involved in the recent update or not. Then
\[
{\bm w}_i^{(k)} = \frac{c_i^{(k)}}{\tau_i} \hat{\bm w}_i^{(k)} + \frac{\tau_i - c_i^{(k)}}{\tau_i} \bar{\bm w}_i^{(k)}.
\]

Here, we can study the model from at least two aspects: \vspace{.2cm}
\begin{enumerate}
\item Under the absence of stubborn agent, is it possible to design ${\bm P}^{(k)}$ such that the convergence to $\frac{1}{|V|} \sum_{j \in V} {\bm w}_j^{(0)}$ can still be reached for each agent\footnote{We may further impose the constraint that we cannot modify the update rule for the reluctant agent.}. For this reason, it is obvious that we can require the agents to exchange \emph{only periodically}, e.g., every ${\rm lcm}(\tau_1,...,\tau_{|V_r|})$ time steps. However, the efficiency of the protocol will be extremely low...\vspace{.2cm}

{\color{red} To: What I am fearing is that it seems to be rather difficult to find a consensus protocol that combat the bias we observed under the reluctant agent model. Thus far, I have not heard of such techniques. Yet, it is certainly a good contribution if there is such; or another way out is to analyze the behavior of opinion dynamics.}\vspace{.2cm}

Another issue that intrigues me is the application of the above model. Idealistically, I want this model to accurately model some `real-world' phenomena. For example, we can think of an election campaign when some voters are uncertain about their vote, I believe that these voters tend to be more `reluctant' than the others. In this case, the position of them in the network may lead to some bias in the network consensus and thus affecting the election results. \vspace{.2cm}

\item Assume that the stubborn agents are adversarial and they want to influence the network by spreading their (unanimous) opinion. Now, when the placement of the normal and reluctant agents is fixed, is it possible \emph{optimally} place the stubborn agent such that the idea held by $V_s$ is most effectively spreaded over the network. Specifically, in this case, is the existence of reluctant agents good for the network?\end{enumerate}


%This is a placeholder for the discrete opinion dynamics model.

\subsection{Previous works and references} 
A good survey paper for opinion dynamics can be found in \cite{Fagnani2014}. The inclusion of stubborn agents in opinion dynamics has actually been studied in \cite{Acemoglu2013} before. We can regard the inclusion of the reluctant agents as an extension to the previous works. In addition, it seems that the reference \cite{Das2014} has conducted simulation based studies in a scale that is sufficiently interesting.  \vspace{.2cm}

However, despite that the proposed model is new, part of my goal is different from most of the previous works. Instead of analyzing the behavior of the opinion dynamics, my endeavor is to design the protocol that leads to consensus despite the existence of reluctant agents.

\section{Applications: attacking the BitCoin network}

{\color{red} To: Before we proceed, there are some concerns that I have about regarding the existence of an attacker. In the BitCoin paper, it was already suggested that an attacker has a choice between getting the reward by mining faster than the others or simply to forge a transaction. It seems to me that the possibility of having an attacker is low. I wonder how we should address this issue.}\vspace{.2cm}

The next question is how to apply opinion dynamics on the attacks on BitCoin. First of all, I believe that the attack model falls into the class of discrete opinion dynamics. In which the attacker and the normal users compete to conquer the entire network (while the attackers are actually `stubborn'). One problem that we can study is where to place the attackers such that the attack will always succeed. \vspace{.2cm}

\bibliographystyle{IEEEtran}
\bibliography{paper}


\end{document}
